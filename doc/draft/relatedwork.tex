\section{Related work}

To the best of our knowledge there have been no works that treat
synchronization of transactions from a principalled, corrective
manner. We now summarize other recent programming methodologies in
exploiting parallelism within transactional systems. These works
explore different ways to improve performance by either reducing
aborts and/or the extent to which a transaction rolls back. As a clear
indication of distinction, note that our corrective approach has no
notion of rollback.

Checkpointing in transactional boosting~\cite{spaa08a} and nested
transactions~\cite{ont,beeri} attempts to, in the event of a conflict,
avoid wasted work done by transactions by only partially
aborting. Elastic transations~\cite{FGG:DISC09} similarly avoid wasted
work by splitting into multiple pieces.  The Janus algorithm for lazy
optimistic synchronization~\cite{TMFS:PLDI12}, \red{does...}
Golan-Gueta \emph{et al.}~\cite{GRSY:PLDI13} on synchronization with
foresight \red{does...}  Tripp \emph{et al.}~\cite{TKS:OOPSLA13} show
how nondeterminism in data-structure specifications offers an
opportunity for increased parallelism.



Our work employs \emph{static analysis} to generate information that
can be used at runtime to improve performance. Some have explored
others ways of using static information, e.g., to infer locking
schemes for transactions~\cite{gulwani}.

Bill McCloskey's POPL'05/06 work
Peter Hawkins' works
Guy's work

%%% Local Variables:
%%% mode: latex
%%% TeX-master: "paper"
%%% End:
