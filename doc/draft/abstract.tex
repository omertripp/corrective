We describe a novel use of abstract interpretation in which the
abstract domain informs a runtime system to correct synchronization
failures. To this end, we first introduce a novel synchronization paradigm,
dubbed \emph{corrective synchronization},
that is a generalization of existing approaches to ensuring serializability.
%
Specifically, the correctness of multithreaded execution need not be
enforced by previous methods that either reduce parallelism
(pessimistic) or roll back illegal thread interleavings (optimistic);
instead inadmissible states can be altered into admissible ones.
%
In this way, the effects of inadmissible interleavings can be
compensated for by modifying the program state as a transaction
completes, while accounting for the behavior of concurrent
transactions.
%The system automatically compensates, if
%necessary, for the effects of inadmissible interleavings by 
%
We have proved that corrective
synchronization is serializable and give conditions under which
progress is ensured. Next, we describe an abstract
interpretation that is able to compute these valid serializable
post-states w.r.t. a transaction's entry state by computing
an underapproximation of the serializable intermediate (or final) states
as the fixpoint solution over an interprocedural control-flow graph.
%
These abstract states
inform a runtime system that is able to perform state correction
dynamically. We have instantiated this setup to clients of a Java-like
Concurrent Map data structure to ensure safe composition of map
operations. Finally, we report early encouraging results that the
approach competes with or out-performs previous pessimistic or
optimistic approaches.
