\section{Introduction}\label{Se:intro}

Concurrency control is a hard problem. While some thread interleavings are admissible (if they involve disjoint memory accesses), there are certain interleaving scenarios that must be inhibited to ensure serializability. The goal is to automatically detect with high precision and low overhead the inadmissible interleavings, and avoid them.  

Toward this end, there are currently two main synchronization paradigms:
\begin{itemize}
	\item \textit{Pessimistic synchronization}: In this approach, illegal interleaving scenarios are avoided conservatively by blocking the execution of one or more of the concurrent threads until the threat of incorrect executions has gone away. Locks, mutexes and semaphores are all examples of how to enforce mutual exclusion, or pessimistic synchronization.
	\item \textit{Optimistic synchronization}: As an alternative to proactive, or pessimistic, synchronization, optimistic synchronization is essentially a reactive approach. The concurrency control system monitors execution, such that when an illegal interleaving scenario arises, it is detected as such and appropriate remediation steps are taken. A notable instance of this paradigm is transactional memory (TM) \cite{DBLP:conf/isca/HerlihyM93}, where the system logs memory accesses by each of the threads, and is able to reverse the effects of a thread and abort/restart it.
\end{itemize}

\paragraph{Motivation.} The pessimistic approach is useful if critical sections are short, there is little available concurrency, and the involved memory locations are well known \cite{AndiKleen}. Optimistic synchronization is most effective when there is a high level of available concurrency. An example is graph algorithms, such as Boruvka, over graphs that are sparse and irregular \cite{KulkarniGalois}.

Beyond these cases, however, there are many other situations of practical interest. As an illustrative example, we refer to the code fragment in Figure \ref{Fi:introMotivating}, extracted from the {\sf dyuproject} project,\footnote{\url{https://code.google.com/archive/p/dyuproject/}} where a shared {\sf Map} object, (pointed-to by) {\sf \_convertors}, is manipulated by method {\sf getConvertor}.\footnote{The code in Figure \ref{Fi:introMotivating} is a slight variation of the original code of {\sf dyuproject} where we simplified the syntax for the sake of clarity}

%\begin{figure}
%	\begin{lstlisting}[numbers=left]
%public Convertor getConvertor(
%      Class cls,boolean create,boolean add) {
%  Convertor convertor = _convertors.get(cls.getName());
%  if(convertor==null && create) {
%    convertor = new Convertor(cls,add);
%    _convertors.putIfAbsent(cls.getName(), convertor); }
%    return convertor; }
%	\end{lstlisting}
%	\caption{\label{Fi:introMotivating}Method {\sf getConvertor()} from class {\sf StandardConvertorCache} in project {\sf dyuproject}}
%\end{figure}

\begin{figure}[t]
  \begin{quote}
	\begin{lstlisting}[numbers=left]
Convertor getConvertor(String name, boolean create) {
  Convertor conv = _convertors.get(name);
  if(conv==null && create) {
    conv = new Convertor(name);
    _convertors.putIfAbsent(name, conv); }
  return conv; }
	\end{lstlisting}
        \end{quote}
	\caption{\label{Fi:introMotivating} {\sf getConvertor()} from class {\sf StandardConvertorCache} in {\sf dyuproject}}
\end{figure}

Assume that different threads invoking this method are all attempting to simultaneously obtain the {\sf Convertor} object related to the same key, which is not yet part of the map. Doing so optimistically would lead to multiple rollbacks (even under boosted conflict detection \cite{ppopp08}, since the operations due to different threads do not commute), and thus poor performance. Mutual exclusion, on the other hand, would block all threads but one until the operation completes, which is far from optimal if {\sf new Convertor()} is an expensive operation.

\paragraph{Corrective synchronization.} In this paper, we take a first step in formulating and exploring a novel synchronization paradigm, which is conceptually different from both the pessimistic and the optimistic approaches. In our approach, dubbed \emph{corrective synchronization}, the correctness of multi-threaded execution is enforced after the fact, similarly to optimistic synchronization, though without rollbacks. Instead, the system automatically compensates, if necessary, for the effects of inadmissible interleavings by rewriting the program state as a transaction completes. This is done while accounting for the behavior of concurrent transactions, so as to guarantee serializability.

To illustrate our approach, we revisit the running example. Consider the following execution history: % in Figure \ref{Fi:motivatingInterleaving}.
	\begin{center}
		\begin{tabular}{r@{}lrcl@{}l}
			[ & & $T_1$: ${\sf get()}/{\sf null}$ & $\rightarrow$ & $T_2$: ${\sf get()}/{\sf null}$  & \\
			& $\hookrightarrow$ & $T_1$: ${\sf if(\ldots)}$ & $\rightarrow$  & $T_2$: ${\sf if(\ldots)}$ & \\ 
			& $\hookrightarrow$ & $T_1$: ${\sf new\ \ldots}/o_1$ & $\rightarrow$ & $T_2$: ${\sf new\ \ldots}/o_2$ & \\ 
			& $\hookrightarrow$ & $T_1$: ${\sf putIfAbsent()}/{\sf null}$ & $\rightarrow$ & $T_2$: ${\sf putIfAbsent()}/o_1$  & \\ 
			& $\hookrightarrow$ & $T_1$: ${\sf return}\ o_1$ & $\rightarrow$ & $T_2$: ${\sf return}\ o_2$ & ]
		\end{tabular}
		\end{center}
This history is clearly not serializable. In any serializable history, $T_1$ and $T_2$ would return the same {\sf Convertor} instance. Correcting this execution involves the application of two actions to the exit state of $T_2$. First, we point the local variable ${\sf conv}$ to $o_1$, rather than $o_2$. Second, we fix the mapping under ${\sf \_convertors}$ for key {\sf name} in the same way.

%\begin{figure}
%	\begin{center}
%		\footnotesize
%		\begin{tabular}{rrll}
%[ & $T_1$: ${\sf get()}/{\sf null}$, & $T_2$: ${\sf get()}/{\sf null}$,  & \\
%& $T_1$: ${\sf if(\ldots)}$, & $T_2$: ${\sf if(\ldots)}$, & \\ 
%& $T_1$: ${\sf new\ \ldots}/o_1$, & $T_2$: ${\sf new\ \ldots}/o_2$, & \\ 
%& $T_1$: ${\sf putIfAbsent()}/{\sf null}$, & $T_2$: ${\sf putIfAbsent()}/o_1$,  & \\ 
%& $T_1$: ${\sf return}\ o_1$, & $T_2$: ${\sf return}\ o_2$ & ]
%		\end{tabular}
%%		\begin{tabular}{c||c}
%%			$T_1$ & $T_2$ \\
%%			\hline
%%			${\sf \_convertors.get()} / {\sf null}$ &  \\
%%			& ${\sf \_convertors.get()} / {\sf null}$ \\
%%			${\sf if (...)}$ 								   &							\\
%%			& ${\sf if (...)}$ \\
%%			${\sf new\ Convertor()} / o_1$		& \\
%%			& ${\sf new\ Convertor()} / o_2$ \\
%%			${\sf \_convertors.putIfAbsent()} / {\sf null}$ &  \\
%%			& 		${\sf \_convertors.putIfAbsent()} / o_1$ \\
%%			${\sf return}\ o_1$ & \\
%%			& ${\sf return}\ o_2$ \\
%%			\end{tabular}
%			\end{center}
%	\caption{\label{Fi:motivatingInterleaving}Interleaved execution of two instances of the code in Figure \ref{Fi:introMotivating}}
%\end{figure}

Note that the corrective actions above are of a general form, which is not limited to only two threads. For any number of threads, the corrected state would have one privileged thread deciding the return value (i.e., the value of {\sf conv}) for all threads, which would also be the value linked by the key under {\sf \_convertors}. In addition, our experiments suggest the corrective actions are --- relatively speaking --- inexpensive, especially compared to the alternatives of either blocking or aborting/restarting all threads but one.

\paragraph{This paper.} Corrective synchronization, as a concept, opens up a vast space of possibilities for concrete synchronization protocols. In this paper, we take a first step in exploring this space.
%
\blue{We introduce the following proof rule:}
$$
\infer{(T, \mu, \sigma, L) \rightarrow (T,\mu[t \mapsto \mu'(t)], \sigma, L)}{
	t \in T, 
	{\sf ref}\ t \leadsto (T,\mu', \sigma, L)}
$$
Configurations consist of tuples $(T,\mu,\sigma,L)$ where
$T$ is a set of transaction ids,  $\mu$ is a mapping from
transcation id to thread-local state, $\sigma$ is the shared state and
and, following~\cite{Koskinen15}, we use a log of events $L$ to track the
state of the system.
%
Finally, $\textsf{ ref}\ t$ is a helper function that remembers the
system state before transaction $t$ began.
%
The key idea is a form of thread-local correction, whereby a single
thread $t$ can apply a correction and suddenly behave as if it had
internally taken a different path.
%
\blue{This rule is surprisingly simple yet, to the best of our
  knowledge, nothing like this exists in the literature.}

For the sake of concreteness, 
%% In particular,
we focus on concurrent Java-like programs whose shared state is encoded as one or more {\sf ConcurrentMap} instances.
%The motivation for this initial scope is a series of past studies, which indicate that {\sf ConcurrentMap} is commonly used in concurrent applications, but often without sufficient synchronization \cite{oopsla/ShachamBASVY11,issta/ShachamYGABSV14}. We study several such examples from popular open-source code bases, including Tomcat and Gridkit.
%
Two important challenges w.r.t. corrective synchronization that we investigate  are (i) for each thread how to compute these alternative poststates; and (ii) given an incorrect poststate, how to efficiently recover the execution to a correct poststate. 
We tackle these challenges via a novel abstract interpretation, equipped with a specialized abstraction for maps, to derive the correct poststates in relation to a given prestate. 

\red{add explanation of Section 5 here}

\OmerAdded{We also discuss extensions, beyond this initial scope, to other data structures and concurrent execution configurations. These highlight directions for future research on corrective synchronization.} 

%To gain insight into the merit of corrective synchronization, we have created a prototype implementation for our scope of {\sf ConcurrentMap} clients. We compare our approach against both STM and lock-based synchronization across different concurrency levels and workload sizes. The results are encouraging: While both STM and corrective synchronization are significantly better than locks, the relative performance improvement thanks to the corrective approach is twice of that achieved by STM.
%
%We govern our discussion of these challenges by a formal framework with rigorous soundness guarantees, which extends beyond the limited scope of {\sf ConcurrentMap} objects. 
%
%
%Beyond the formal details, we address the question of how to implement corrective synchronization efficiently, such that it incurs low overhead. To this end, we present a solution based on static analysis to derive the correct poststates in relation to a given prestate. We have implemented a version of the analysis for shared {\sf Map} data structures. {\sf Map}s are used to represent the shared state of many Java programs \cite{OhadOOPSLA}, and so this first step toward a comprehensive static analysis for corrective synchronization is already of practical value.
%
%To examine the promise and viability of corrective synchronization, we have experimented with concurrent {\sf Map}-based data structures from popular open-source code bases, such as Tomcat and Gridkit, which were found by previous studies to require additional synchronization \cite{oopsla/ShachamBASVY11,issta/ShachamYGABSV14}. On these benchmarks

%\red{say that the implementation does something very simple (assumes one transaction), yet illustrates something very powerful that is completely novel, add something about the experimental results}

In summary, this paper makes the following principal contributions.
%\begin{enumerate}
  (1) We present an alternative to both the pessimistic and the optimistic synchronization paradigms, dubbed \emph{corrective synchronization}, whereby serializability is achieved neither via mutual exclusion nor via rollbacks, but through correction of the poststate according to a relational prestate/poststates specification.
(2) We provide a formal description of corrective synchronization. This includes a soundness proof as well as a clear statement of limitations.
	(3) We have developed an abstract interpretation to derive the prestate/poststates specification for programs that encode the shared state as one or more concurrent maps. 
%	\item \underline{Implementation and evaluation}: We have created an implementation of corrective synchronization assuming the shared state is represented as a collection of concurrent maps. We present experimental evidence in favor of corrective synchronization, where our subjects are real-world Java applications.
(4) We have developed a runtime system that is able to use
pre/post-state specifications to correct behaviors dynamically.
%
(5) We report encouraging preliminary experimental results on
our prototype implementation.

