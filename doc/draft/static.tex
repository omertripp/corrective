\newcommand{\set}[1]{\mathsf{#1}}
\newcommand{\isSummary}{\set{summary}}
\newcommand{\freshNode}{\set{fresh}}
\newcommand{\heapnode}{\set{HeapNode}}
\newcommand{\variable}{\set{Var}}
\newcommand{\env}{\set{Env}}
\newcommand{\map}{\set{Map}}
\newcommand{\state}{\set{\Sigma}}
\newcommand{\serializedCFGs}{\set{serializedCFGs}}
\newcommand{\iseqclass}{\set{eqClass}}
\newcommand{\warpdestination}{\set{warpDest}}

\section{Static Computation of Warp Destinations}

\subsection{Abstract Domain}
\label{sect:abstractate}

Let $\variable$ and $\heapnode$ be the set of variables and abstract heap nodes, respectively. We suppose that a special \statement{null} value is part of $\heapnode$. Both keys and values are abstracted as heap nodes. As usual with heap abstractions, each heap node might represent one or many concrete nodes. Therefore, we suppose that a function $\isSummary : \heapnode \to \{\true, \false\}$ is provided; $\isSummary(h)$ returns $\true$ if and only if $h$ represents many concrete nodes (that is, it is a summary node). We define by $\env : \variable \to \wp(\heapnode)$ the set of (abstract) environments relating each variable to the set of heap nodes it might point to. A map is represented as a function $\map : \heapnode \to \wp(\heapnode)$, connecting each key to the set of possible values it might be related to in the map. The value $\statement{null}$ represents that the key might not be in the map. For instance, $[n_1 \mapsto \{\statement{null}, n_2\}]$ represents that the key $n_1$ might not be in the map, or it is in the map, and it is related to value $n_2$. An abstract state is a pair made by an abstract environment and an abstract map. We augment this set with a special bottom value $\bot$ to will be used to represent that a statement is unreachable. Formally, $\state = (\env \times \map) \cup \{\bot\}$.

The lattice structure is obtained by the point-wise application of set operators to elements in the codomain of abstract environments and functions. Therefore, the abstract lattice is defined as $\langle \state, \dot{\subseteq}, \dot{\cup} \rangle$, where $\dot{\subseteq}$ and$\dot{\cup}$ represents the point-wise application of set operators $\subseteq$ and $\cup$, respectively.

\paragraph{Running example.} 
Consider the first method in Figure \ref{lst:runningexamplestaticanalysis}. For instance, the abstract state $([\statement{k} \mapsto \{n_1\}], [n_1 \mapsto \{\statement{null}\}])$ represents that the key \statement{key} is not in the map, while $([\statement{k} \mapsto \{n_1\}], [n_1 \mapsto \{n_2\}])$ represents that it is in the map, and it is related to a value. Instead, for the second method, $([\statement{k} \mapsto \{n_1\}, \statement{v} \mapsto n_2], [n_1 \mapsto \{n_2\}])$ represents that \statement{k} is in the map, and it is related to the value pointed by \statement{v}


\subsection{Abstract Semantics}
\label{sect:abstractsemantics}

\begin{figure*}
\[
\begin{array}{ll}
\csemantics{S}{\statement{m.put(k, v)}, (e, m)} = \left\{
\begin{array}{ll}
(e, m[n \mapsto e(\statement{v})]) & \textrm{if } e(\statement{k})=\{n\} \land \neg \isSummary(n)\\
(e, m[n \mapsto m(n) \cup e(\statement{v}) : n \in e(\statement{k})]) & \textrm{otherwise}\\
\end{array}
\right. & (\mathtt{put})\\
\\
\csemantics{S}{\statement{v=m.get(k)}, (e, m)} = (e[\statement{v} \mapsto \bigcup_{n \in e(k)} m(n)], m) & (\mathtt{get})\\
\\
\csemantics{S}{\statement{m.remove(k)}, (e, m)} =  \left\{
\begin{array}{ll}
(e, m[n \mapsto \{\statement{null}\}]) & \textrm{if } e(\statement{k})=\{n\} \land \neg \isSummary(n)\\
(e, m[n \mapsto m(n) \cup \{\statement{null}\} : n \in e(\statement{k})]) & \textrm{otherwise}\\
\end{array}
\right. & (\mathtt{rmv})\\ 
\\
\csemantics{S}{\statement{v = m.putIfAbsent(k, v)}, (e, m)} =  (\pi_1(\csemantics{S}{\statement{v = m.get(k)}, (e, m)}), m') : & \\
\hspace{70pt} 
m'=
\left\{
\begin{array}{ll}
(e, m[n \mapsto e(\statement{v})]) & \textrm{ if } e(\statement{k})=\{\statement{n}\} \land m(n) = \{\statement{null}\}\\
(e, m[n \mapsto m(n) \cup e(\statement{v}) : n \in e(\statement{k}) \land \statement{null} \in m(n)]) & \textrm{ otherwise}\\
\end{array}
\right. & (\mathtt{pIA})\\
\\
\csemantics{S}{\statement{v = new\ Value()}, (e, m)} =  (e[v \mapsto \freshNode(\statement{t})], m)& (\mathtt{new})\\
\\
\csemantics{S}{\statement{v = new\ Value()}, (e, m)} =  (e[v \mapsto \{\statement{null}\}], m)& (\mathtt{null})\\
\\
\csemantics{S}{\statement{if(b)\ s_1;\ else\ s_2}, (e, m)} =  \csemantics{S}{\statement{s_1}, \csemantics{B}{\statement{b}, (e, m)}} \dot{\cup} \csemantics{S}{\statement{s_2}, \csemantics{B}{\statement{! b}, (e, m)}} & \statement{(if)}\\
\\
\csemantics{S}{\statement{while(b)\ s_1;}, (e, m)} = \csemantics{B}{\statement{! b}, (e_1, m_1)} : 
(e_1, m_1) = \mathit{lfp}^{\dot{\subseteq}}_{\bot} \lambda (e', m') . (e, m) \dot{\cup} \csemantics{S}{\statement{s_1}, \csemantics{B}{\statement{b}, (e', m')}}& \statement{(while)}\\
\\
\csemantics{S}{\statement{s_1;\ s_2}, (e, m)} =  \csemantics{S}{\statement{s_2}, \csemantics{S}{\statement{s_1}, (e, m)}} & \statement{(cnc)}\\
\\
\csemantics{B}{\statement{x==null}, (e, m)} = \left\{
\begin{array}{ll}
(e[\statement{x} \mapsto \{\statement{null}\}], m) & \textrm{if } \statement{null} \in e(\statement{x})\\
\bot & \textrm{otherwise}\\
\end{array}
\right. & (\mathtt{null})\\
\\
\csemantics{B}{\statement{!x==null}, (e, m)} = \left\{
\begin{array}{ll}
(e[\statement{x} \mapsto e(\statement{x}) \setminus \{\statement{null}\}], m) & \textrm{if } \exists n \in \heapnode : n \neq \statement{null} \land n \in e(\statement{x})\\
\bot & \textrm{otherwise}\\
\end{array}
\right. & (\mathtt{!null})\\
\\
\csemantics{B}{\statement{m.containsKey(k)}, (e, m)} = \left\{
\begin{array}{ll}
\bot & \textrm{if } \forall n \in e(\statement{k}) : m(n)=\{\statement{null}\}\\
(e, m[n \mapsto m(n)\setminus\{\statement{null}\}]) & \textrm{if } e(\statement{k}) = \{n\} \land \not \isSummary(n) \land m(n) \neq \{\statement{null}\}\\
(e, m) & \textrm{otherwise}\\
\end{array}
\right. & (\mathtt{cntK})\\
\\
\csemantics{B}{\statement{! m.containsKey(k)}, (e, m)} = \left\{
\begin{array}{ll}
\bot & \textrm{if } \forall n \in e(\statement{k}) : \statement{null} \notin m(n)\\
(e, m[n \mapsto \{\statement{null}\}) & \textrm{if } e(\statement{k}) = \{n\} \land \not \isSummary(n) \land \statement{null} \in m(n)\\
(e, m) & \textrm{otherwise}\\
\end{array}
\right. & (\mathtt{!cntK})\\
\end{array}
\]
\caption{Formal definition of the abstract semantics}
\label{fig:abstractsemantics}
\end{figure*}
Figure \ref{fig:abstractsemantics} formalizes the abstract semantics of statements and Boolean conditions, that, given an abstract state (as defined in Section \ref{sect:abstractate}) and a statement or Boolean condition of the language introduced in Section \ref{sect:language}, returns the abstract state resulting from the evaluation of the given statement on the given abstract state. We focus the formalization on abstract states in $\env \times \map$, since in case of $\bot$ the abstract semantics always returns $\bot$ itself.

\statement{(put)} relates \statement{k} to \statement{v} in the map. In particular, if \statement{k} points to a unique concrete node, it performs a so-called strong update, overwriting previous values related with \statement{k}. Otherwise, it performs a weak update by adding to the previous values the new ones. \statement{(get)} relates the assigned variable \statement{v} to all the heap nodes of values that might be related with \statement{k} in the map. Note that if \statement{k} is not in the map, then the abstract map $m$ relates it to a \statement{null} node, and therefore this value is propagated to \statement{v} then calling \statement{get}, representing the concrete semantics of this statement. Similarly to \statement{(put)}, \statement{(rmv)} removes \statement{k} from the map (by relating it to the singleton $\{\statement{null}\}$) iff \statement{k} points to a unique concrete node. Otherwise, it adds the heap node \statement{null} to the heap nodes related to all the values pointed by \statement{k}. \statement{(pIA)} updates the map like \statement{(put)} but only if the updated key node might have been absent, that is, when $\statement{null} \in m(n)$. \statement{new} creates a new heap node through $\freshNode(t)$ (where $t$ is the identifier of the transaction performing the creation), and assigns it to \statement{v}. The number of nodes is kept bounded by parameterizing the analysis with an upper bound $i$ such that (i) the first $i$ nodes created by a transaction are all concrete nodes, and (ii) all the other nodes are represented by a summary node. Instead, \statement{(null)} relates the given variable to the singleton $\{\statement{null}\}$.
Rules \statement{(if)}, \statement{(while)}, and \statement{(cnc)} define the standard abstract semantics of \statement{if}, \statement{while}, and concatenation statements.
The abstract semantics on Boolean conditions produces $\bot$ statements if the given Boolean condition cannot hold on the given abstract semantics. Therefore, \statement{(null)} returns $\bot$ if the given variable \statement{x} cannot be \statement{null}, or a state relating \statement{x} to the singleton $\{\statement{null}\}$ otherwise. Vice-versa, \statement{(!null)} returns $\bot$ if \statement{x} can be only null, or a state relating \statement{x} to all its previous values except \statement{null} otherwise.
Similarly, \statement{(cntK)} returns $\bot$ if the given key \statement{k} is surely not in the map, it refines the possible values of \statement{k} if it is represented by a concrete node, or it simply returns the entry state otherwise. Vice-versa, \statement{(!cntK)} returns $\bot$ if \statement{k} is surely in the map.

\paragraph{Running example.}
Consider again the first method in Figure \ref{lst:runningexamplestaticanalysis}. When we start from the abstract state  $([\statement{k} \mapsto \{n_1\}], [n_1 \mapsto \{\statement{null}\}])$ (representing that \statement{k} is not in the map), we obtain the abstract state $\sigma = ([\statement{k} \mapsto \{n_1\}, \statement{result} \mapsto \{null\}], [n_1 \mapsto \{\statement{null}\}])$ after the first statement by rule \statement{(null)}. During the following computation of rule \statement{(if)}, we consider:
\begin{enumerate}
	\item when the Boolean condition \statement{map.containsKey(k)} holds. When applying rule \statement{(cntK)} on $\sigma$ we obtain $\bot$ since the node pointed by \statement{k} is related to the singleton $\{\statement{null}\}$ in the map, representing that the map does not contain the key \statement{k}; and
	\item when \statement{! map.containsKey(k)} holds. Rule \statement{(!cntK)} applied to $\sigma$ returns $\sigma$ itself, since \statement{k} is in relation only with \statement{null} in the map.
\end{enumerate}
\statement{(if)} returns the upper bound of the two resulting states, that is $\bot \dot{\cup} \sigma = \sigma$, and the value pointed by \statement{result} (that is, \statement{null}) is returned.
Therefore, our analysis computes on this example that, when the key is not in the map in the entry state, the method returns \statement{null} and does not modify the map.

\subsection{Over-approximating Serialized Executions}
We now formalize how we compute an approximation of serialized executions relying on the abstract semantics defined in Section \ref{sect:abstractsemantics}. Our system receives as input two types of transactions (e.g., the two methods in Figure \ref{lst:runningexamplestaticanalysis}), and returns a mapping from possible abstract entry states, to sets of possible exit states.

In particular, given two transactions \statement{t1} and \statement{t2}, we build up a control flow graph that represents possible serialized executions of many instances of the two transactions. In order to build up this serialized control flow graph, the local variables of transactions are rename. For instance, we can build up a serialized CFG where an instance of transaction \statement{t1} (called \statement{t1_1}) is followed by a loop representing a (possible unbounded) sequence of executions of other instances of the same transaction represented by a loop of a \emph{summary} instance of \statement{t1} (called \statement{t1_n}), and this is followed by a similar structure for transaction \statement{t2} (generating transactions \statement{t2_1} and \statement{t2_n}).

When applied to these serialized control flow graphs, the static analysis engine produces exit states that are possible results of serialized executions. Given two transactions, we represent by $\serializedCFGs(\statement{t1}, \statement{t2})=\set{T}$ the function that returns a set of serialized CFGs.

\subsection{Extracting Possible Warping States}
As we discussed in \pietrotodo{XXX}, we need to compute warping that, given an entry state representing an observational equivalence class, is in an equivalence class that is reachable through a serialized execution. However, an abstract state in $\state$ might represent concrete states that are in different equivalence classes. For instance $([\statement{k} \mapsto \{n_1\}], [n_1 \mapsto \{\statement{null}, n_2\}])$ represents both that  \statement{k} is (if $n_1$ is related to $n_2$ in the abstract map) or is not (when $n_1$ is related to $\statement{null}$). This abstract state therefore might concretize to states belonging to different equivalence classes, and it cannot used to define a warping destination.

Therefore, we define a predicate $\iseqclass : \state \to \{\true, \false\}$ that, given an abstract state, holds iff it represents concrete states all in the same equivalence class. Formally,
\[
\begin{array}{c}
\iseqclass(e, m)\\
\Updownarrow\\
\forall \statement{x} \in \dom{e} : |e(\statement{x})|=1 \land e(\statement{x})=\{n_1\} \land \neg \isSummary(n_1)\\
\forall n \in \dom{m} : |m(n)|=1 \land m(n)=\{n_2\} \land \neg \isSummary(n_2)\\
\end{array}
\]
\pietrotodo{Prove that if $\iseqclass(e,m)$ then all the concretized states from $(e, m)$ are in the same equivalence class}


Given two transactions \statement{t1} and \statement{t2}, we build up a set of possible entry states $S$ such that $\forall (e, m) \in S : \iseqclass(e, m)$. We then compute the exit states for all the possible serialized CFGs and entry states, and we filter our only the ones that represents states in the same equivalence class. The results are represented as a function that relates each entry state to a set of possible exit states. This is formalized by the following $\warpdestination$ function.


\[
\begin{array}{l}
\warpdestination(\statement{t1}, \statement{t2}, \set{S}) =\{(e', m'):\\
\hspace{10pt} \exists (e, m) \in \set{S}, \exists \statement{s} \in \serializedCFGs(\statement{t1}, \statement{t2}) :\\
\hspace{20pt} (e', m') \in \csemantics{S}{\statement{s}, (e, m)} \land \iseqclass(e', m')\}\\
\end{array}
\]

\pietrotodo{Not yet explained how we compute the possible entry states}

\newcommand\ts{\bar{t}}


\subsection{Concrete state}
Let $\Sigma = \gamma(\Ts \times G)$ be the concrete states, denoted $\sigma = (\ts,g)$.

\newcommand\Pietrot{{\cal P}ietro}

Discover a $\Pietrot : \hat{\Sigma} \rightarrow \hat{\Sigma}
\rightarrow \wp({\hat{\Sigma}})$, representing the current abstract
state, the reference abstract state, and a set of possible destination
abstract states.



%\newcommand\llangle{\langle\!\langle}
%\newcommand\rrangle{\rangle\!\rangle}
