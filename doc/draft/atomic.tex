\section{Atomic Transition Systems}
\label{apx:atomic}


\begin{figure}
\figbox{\footnotesize
(a) {\bf Atomic Machine Big Step Transaction Rules} $\BigStep$
$$
\infer[\text{\sc BSFin}]{
  (c,\sigma),\OPL \BigStep c,\OPL
}{\nothing{c}}
\qquad
\infer[\text{\sc BSStep}]{
  (c,\sigma),\OPL_1 \BigStep \sigma'',\OPL_2
}{
  \deduce{
    (c_2,\sigma'),
    \OPL\cdot[\langle m,\sigma,\sigma' \rangle]
   \BigStep \sigma'',\OPL_2
 }{
   \deduce{\OPL_1 \allows \langle m,\sigma,\sigma' \rangle}
   {\step{c}{m}{c_2}}
 }
}
$$
(b) {\bf Atomic Machine Rules} $\xrightarrow{a}$
$$
\infer[\text{\sc AFin}]{ 
  \As_1 \cdot (c,\sigma) \cdot \As_2, G  \xrightarrow{a}
  \As_1 \cdot \As_2, G 
}{
  \nothing{c}
}
$$
$$
\infer[\text{\sc AFork}]{ 
  \As_1 \cdot (c,\sigma) \cdot \As_2, G  \xrightarrow{a}
  \As_1 \cdot (c_2,\sigma) \cdot (c',\sigma) \cdot \As_2, G 
}{
  \tstep{c_1}{\fork{c}}{c_2}
}
$$
$$
\infer[\text{\sc ALocal}]{ 
  \As_1 \cdot (c,\sigma) \cdot \As_2, G  \xrightarrow{a}
  \As_1 \cdot (c',\sigma') \cdot \As_2, G 
}{
  \tstep{c}{\local{R}}{c'} & R\ \sigma\ \sigma'
}
$$
$$
\infer[\text{\sc ATxn}]{ 
  \As_1 \cdot (c_1,\sigma) \cdot \As_2, G  \xrightarrow{a}
  \As_1 \cdot (c_2,\sigma') \cdot \As_2, G'
}{
  \tstep{c_1}{\tx{c'}}{c_2} &
  (c',\sigma),G \BigStep \sigma',G'
}
$$
}
\caption{\label{fig:atomic} Atomic semantics of concurrent threads.}
\end{figure}



We next define a simple atomic semantics $\xrightarrow{a}$,
in which transactions are executed instantly, without interruption
from concurrent threads.

The $\xrightarrow{a}$ rules {\sc AFin, AFork, ALocal} showin in 
Figure~\ref{fig:atomic}(b) are similar to
their counterparts in $\xrightarrow{u}$.
%
However, the {\sc ATxn} rule says that if thread executing code $c_1$ can reduce to a 
transaction $\tx{c'}$, then the transaction $c'$ is executed
atomically by the big step rules $\BigStep$ described next.

Figure~\ref{fig:atomic}(a) illustrates the
big step semantics $\BigStep$, which uses $\stept$ and $\nothing{}$ 
(rules {\sc BSStep} and {\sc BSFin}, respectively). These rules
scan through the nondeterminism in $\tx{c}$ to find a next operation
name $m$ or a path to $\skipt$ denoting the end of the transaction.
%
{\sc BSStep} can be taken provided that the operation $\langle m,\sigma,\sigma'\rangle$ is
permitted and that $(c_2,\sigma')$ can be
entirely reduced to $(\sigma'',\OPL_2)$.
%  Note that there a new operation identifier $\opid_1$
% is introduced. We require $\opid_1$ to be globally unique
% (formalization of \textsf{fresh} is omitted for lack of space).

%\red{explain bsstep. we go from $\OPL_1$ to $\OPL_2$ .... bsfin
%  basically copies the $\OPL$ from left of $\Downarrow$ to the right}

