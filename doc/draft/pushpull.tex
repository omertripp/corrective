\newcommand\STEP{{\sc step}}
\newcommand\BEGIN{{\sc begin}}
\newcommand\LOCAL{{\sc local}}
\newcommand\FIN{{\sc fin}}
\newcommand\FORK{{\sc fork}}

\subsection{The \PUSH{}/\PULL{} model}

We briefly summarize the \PUSH{}/\PULL{} model~\cite{KP:PLDI15}. This model consists of
seven rules which we will use for both intuitive descriptions of the
Warping protocol, and in the formal proof of serializability.
The \PMPY{} model is a log-based transition relation
$\Ts,G \pmpyreduce \Ts', G'$ that reduces
a list of threads $\Ts : \textsf{list } (c\times \lstack
\times L)$ and a global log $G$ to $\Ts',G'$.
%
There is a per-thread local log $L$ and a global log $G$.


The rules characterize how a given transaction moves forward/backward
locally, how it may share its effects into the shared view (\PUSH),
and bring the effects of other transactions into its local view
(\PULL). The rules are:
\begin{itemize}
\item \APPLY: A thread reduces its (nondeterminisitc) code down to a
  next-reachable operation $\op$ (with continuation code). This
  operation $\op$ is applied to the local log but is not yet shared
  into the global log. It is yet ``unpushed.''

\item \UNAPP: This rule undoes the most recent \APPLY, restoring the
  previous code and local state, and discarding the most recent
  operation from the local log.

\item \PUSH: A transaction shares its effects with the global view by
  copying an $\op$ from its local log and appending it to the global
  log. Locally, $\op$ is considered ``pushed'' and, in the global log,
  it is marked as ``uncommitted.''
%
  This rule can only be taken if all uncommitted operations in the
  shared log can commute with (move to the right of) $\op$.
%
  Moreover, it is possible to \PUSH{} operations to the shared log in
  a different order than they appear in the local log (provided that
  each pushed $\op$ commutes with its earlier unpushed neighbors in the local log).

\item \UNPUSH: An operation $\op$ that has been \PUSH{}ed to the
  shared log can be \UNPUSH{}ed by swapping the flag from ``pushed''
  to ``unpushed'' and removing the corresponding global log entry for
  $\op$. $\op$ can only be \UNPUSH{}ed if it commutes with the
  subsequently pushed operations in the log.

\item \PULL:
%
  Transactions can learn about the published effects of other
  transactions by \PULL{}ing operations from the global log into their
  local logs. Provided it has not alreadty been pulled, an operation
  $\op$ can be copied from the global log, appended to the local log,
  and marked as ``pulled.'' A transaction does not need to \PULL{} in
  shared-log-chronological order.

\item \UNPULL: (description omitted for lack of space)

\item \CMT: \red{todo}
%  and local state manipulations (\LOCAL
\end{itemize}
%The full descriptions of these rules are given in Appendix~\ref{apx:bigfig}.
Note that each of the above rules (except for \CMT) have an opposite rule.
%
Other rules include transaction initiation (\BEGIN),
thread completion (\FIN) and forking (\FORK).
As shown previously~\cite{KP:PLDI15}, each rule comes with a few
criteria which, all told, ensure serializability and---when
appropriately restricted---opacity.




%%% Local Variables:
%%% mode: latex
%%% TeX-master: "paper"
%%% End:
