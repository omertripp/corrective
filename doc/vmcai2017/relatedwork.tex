\section{Related work}
To our knowledge, existing solutions to the problem of correct synchronization assume either the ability to prevent bad interleavings or the ability to roll back execution. We focus our survey of related research on solutions for optimizing the rollback mechanism, and also discuss works on synchronization synthesis backed by static program analysis and on merging state mutations by concurrent threads.

%\paragraph{Rollback optimizations.}
There are two main optimizations to \emph{decrease rollback overhead}:
reducing either abort rate or the extent to which a conflicted
transaction rolls back. Different solutions have been proposed in
each direction~\cite{ppopp/HerlihyK08,Galois,TYFS:OOPSLA11}.
Others leverage available nondeterminism
\cite{TKS:OOPSLA13}. None of these approaches perform corrective
synchronization.
%
%Conflict detection is primarily based on violations of disjoint parallelism. That is, if two transactions perform simultaneous read/write or write/write accesses to the same memory location, then they are deemed conflicting. 
%
%An effective approach to mitigate false conflicts is to leverage the built-in guarantees of concurrent data structures \cite{ppopp/HerlihyK08,Galois,TYFS:OOPSLA11}. As an example, two {\sf ConcurrentHashMap} instances that perform {\sf put} operations involving different keys may still appear to conflict as they both access {\sf size}.
%
%Transactional boosting~\cite{ppopp/HerlihyK08} is a systematic methodology to specify the behavior of linearizable data-structure operations in terms of their semantic footprint. The Galois system \cite{Galois}, follows a similar approach with emphasis on the {\sf Graph} ADT. The Hawkeye tool~\cite{TYFS:OOPSLA11} detects impediments to parallelization accurately by mapping between the concrete and abstract representations of the data structure (albeit not its operations).
%
%In another study, Tripp et al.~\cite{TMFS:PLDI12} mine commutative behaviors involving \emph{multiple} operations out of execution traces of the program that appear to conflict when viewed at the granularity of individual operations. These are then translated into logical conditions used to avoid spurious conflicts in lazy STM.
%
%Another approach to reduce conflict is to leverage available nondeterminism \cite{TKS:OOPSLA13}. If a transaction can choose between nondeterministic behaviors (e.g., selecting among different paths between two graph nodes), then conflict is potentially reduced by directing the transaction toward a path where interference with other threads is less likely.
%
A well-known solution to restrict the extent to which a transaction rolls back is checkpointing \cite{spaa08a,Egwutuoha:2013}
%Checkpointing introduces the notion of a partial abort, where the transaction resumes from some intermediate point rather than fully resetting its execution and effects.
or nested transactions \cite{ont,beeri}.
%achieve a similar effect in that only the nested transaction, but not its parent transaction, aborts and restarts. Yet another mechanism to reduce rollback overhead is
Elastic transactions~\cite{FGG:DISC09} avoid wasted work by splitting into multiple pieces.  
%
The Push/Pull model \cite{KoskinenP15} also uses local/shared logs, and is flexible enough to express rollback-based transactions but nor corrective synchronization.
%. However, corrective synchronization is not expressible in Push/Pull. We introduce a new rule, {\sf corr}, and treat {\sf cmt} differently with the {\sf serpref} function.

In our solution, static analysis is used to identify admissible shared-state configurations to correct to from a given input state. Multiple past works on synchronization synthesis have also \emph{relied on static analysis}, albeit for the extraction of other types of information. 
%
Golan-Gueta et al.~\cite{GRSY:PLDI13} utilize static analysis to compute a conservative approximation of the possible actions that a transaction may perform in its future from a given intermediate point. This still pessimistic approach enables more granular synchronization compared to the worst-case assumption that the transaction may perform any action in its future.
%
Autolocker \cite{popl/McCloskeyZGB06} applies static analysis to determine a correct locking policy.
%, free of deadlocks and race conditions, given (i) a specification of pessimistic atomic sections and (ii) annotations that connect locks to shared sections. The analysis, encoded as a type system, guarantees deadlock and race freedom.
%
Hawkins et al. \cite{HawkinsAFRS12}
%propose a system that transforms a program consisting, in part, of concurrent relations into a program where the relations are represented as concrete concurrent data structures, as well as locks, that together
ensure correct synchronization by construction.
%They guarantee serializability and deadlock freedom.
%
Prountzos et al. \cite{PrountzosMPM11} optimize the Galois system \cite{Galois} via static shape analysis \cite{MoolyToplas}.
%They apply the TVLA system to identify fail-safe points during the transaction's execution, beyond which neither backup nor conflict detection are necessary. 

Finally, we note solutions based on merging, or combining, the effects of concurrent threads. Burckhardt and Leijen propose \emph{concurrent revisions} \cite{ConcRevisionsSPAA}, which takes its inspiration from version control systems. The idea is to specify a (custom) merge function, based on a revision calculus defined by the authors, such that concurrent state mutations by different threads can be reconciled in a deterministic manner. Somewhat similarly, Hendler \emph{et al.}~introduce \emph{flat combining} \cite{FlatCombiningSPAA}. The idea is to synchronize concurrent accesses to a shared data structure $D$ by having threads post their updates to $D$ into a common list as thead-local records, where a single thread at a time acquires the lock on $D$, combines and applies the updates, and writes the results back to the threads' request fields in the associated publication records. Contrary to these two paradigms, our approach builds on thread-level state correction. This bypasses the need for a coarse-grained lock, as in flat combining, and --- unlike concurrent revisions --- ensures serializability. 


%This gives rise to two optimizations: (i) the transaction does not need to back up modified data, and (ii) redundant conflict checking can be eliminated. In Proznos et al.'s solution, the shape analysis, which is parametric in essence, is instantiated via a predicate discovery algorithm that exploits common patterns of data-structure usage.




%%% Local Variables:
%%% mode: latex
%%% TeX-master: "paper"
%%% End:
